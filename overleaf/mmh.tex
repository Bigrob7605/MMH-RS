% mmh.tex — Appendix: Multi-epoch Meta-Hash (MMH) Recursion Framework% To be included via % mmh.tex — Appendix: Multi-epoch Meta-Hash (MMH) Recursion Framework% To be included via % mmh.tex — Appendix: Multi-epoch Meta-Hash (MMH) Recursion Framework% To be included via % mmh.tex — Appendix: Multi-epoch Meta-Hash (MMH) Recursion Framework% To be included via \input{mmh.tex} before the bibliography.\clearpage\appendix% Best-practice: improve line breaking in paragraphs\sloppy% Fix section numbering for appendix\renewcommand{\thesection}{A\arabic{section}}\section{Multi-epoch Meta-Hash (MMH) Recursion Framework}\label{app:MMH}The Multi-epoch Meta-Hash (MMH) framework provides a deterministic algorithm mapping a 128-bit binary seed $S$ to three quantized late-Universe cosmological observables:\[  \left(z_{\mathrm{jump}},\, r_{\mathrm{dip}},\, \ell_{\mathrm{CIB}}\right) = \mathrm{MMH}(S)\]This mapping is fully specified, reproducible, and reference code is available at \url{https://github.com/Bigrob7605/MMH}.\subsection{Seed Expansion via SHA-256}Given $S \in \{0,1\}^{128}$, form a 256-bit hash:\[  H = \mathrm{SHA256}(S) \in \{0,1\}^{256}\]Let $H = b_1b_2 \ldots b_{256}$ (bits). Partition as:\[  \begin{aligned}    C_1 &= b_1 \ldots b_{84} \\    C_2 &= b_{85} \ldots b_{168} \\    C_3 &= b_{169} \ldots b_{256}  \end{aligned}\]Each chunk $C_i$ is treated as a big-endian integer in $[0,\,2^{|C_i|}-1]$.\subsection{Quantization to Physical Observables}Map each chunk $C_i$ to a real observable $X$ by:\[  X = X_{\min} + \frac{\mathrm{int}(C_i)}{2^{|C_i|}-1}\,(X_{\max} - X_{\min})\]where the $X_{\min}$ and $X_{\max}$ are set by the physical search windows:\begin{itemize}  \item \textbf{Quantized Expansion Jump:}    \[      z_{\rm jump} = 0.0723 + \frac{C_1}{2^{84}-1} \times 0.0028    \]    With corresponding $\Delta H_0 = 5.73 \pm 0.44~\mathrm{km\,s^{-1}\,\mathrm{Mpc}^{-1}}$.  \item \textbf{Galaxy Clustering Dip:}    \[      r_{\rm dip} = 153.2 + \frac{C_2}{2^{84}-1} \times 1.9~\mathrm{Mpc}\,h^{-1}    \]    Predicts a $3.4\%$ deficit in $\xi(r)$ at this scale.  \item \textbf{CIB--H$_0$ Cross-Correlation:}    \[      \ell_{\rm CIB} = 197 + \frac{C_3}{2^{88}-1} \times 4    \]    Yields a $2.4\sigma$ CIB $\times$ H$_0$ signal at this multipole.\end{itemize}\textit{Parameter ranges reflect plausible late-Universe anomaly windows based on recent Roman, DESI, and SPHEREx forecasts. The hash partitioning ensures full coverage of each target interval.}\subsection{Reference Python Implementation}For reproducibility, the following pseudocode (Python 3.11+, using \texttt{hashlib} and \texttt{bitstring} for clarity):\begin{lstlisting}[language=Python, frame=single, basicstyle=\ttfamily\footnotesize, numbers=left, numberstyle=\tiny, stepnumber=1, numbersep=5pt, caption=MMH recursion framework implementation, label=lst:mmh_impl]# Python 3.11+; requires bitstring.Bits moduleimport hashlibimport bitstringdef MMH(seed_128bit_hex):    """    Multi-epoch Meta-Hash recursion framework.        Args:        seed_128bit_hex (str): 128-bit hex string (e.g., '0x7f3a2c9e45af01b6da2d4316a2b0e5d1')        Returns:        tuple: (z_jump, r_dip, ell_CIB) with quantized predictions    """    # Convert seed to bytes and hash via SHA-256    seed_bytes = bytes.fromhex(seed_128bit_hex[2:])  # drop '0x'    h256 = hashlib.sha256(seed_bytes).digest()    bits = bitstring.Bits(bytes=h256)        # Extract chunks    C1 = bits[0:84].uint    C2 = bits[84:168].uint    C3 = bits[168:256].uint        # Quantized predictions    z_jump  = 0.0723 + (C1/(2**84-1)) * 0.0028    r_dip   = 153.2  + (C2/(2**84-1)) * 1.9    ell_CIB = 197    + (C3/(2**88-1)) * 4        return round(z_jump,4), round(r_dip,1), round(ell_CIB,0)# Example usage:# MMH("0x7f3a2c9e45af01b6da2d4316a2b0e5d1")# Returns: (0.0723, 153.2, 197)\end{lstlisting}\textbf{Example:}  Input seed: \texttt{0x7f3a2c9e45af01b6da2d4316a2b0e5d1}  Output:\[  z_{\rm jump} \approx 0.0723, \quad r_{\rm dip} \approx 153.2, \quad \ell_{\rm CIB} \approx 197\]\subsection{Rationale and Limitations}The MMH procedure is intentionally minimal, transparent, and easy to reproduce—anyone can re-derive the results. The specific mapping windows reflect prior known anomalies and test regions, but the use of a deterministic cryptographic hash (SHA-256) is for maximal auditability, not as a claim about the physical universe's mechanism.\textbf{Important Disclaimer:} The use of a cryptographic hash is strictly for reproducibility and coverage—not a claim about physical cosmology. SHA-256 is chosen for its deterministic properties and widespread availability, not because it has any physical significance. This framework is a mathematical falsifiability exercise designed for explicit testing, not a theory of cosmic microphysics.This framework is agnostic to the true microphysics: its success or failure is decided solely by the appearance (or non-appearance) of the predicted signatures in Roman, DESI, and SPHEREx data.\textit{Full details, data, and cross-check scripts are versioned in the accompanying public repository.}\subsection{Computational Requirements}The MMH framework requires:\begin{itemize}  \item Python 3.11 or higher  \item \texttt{hashlib} (standard library)  \item \texttt{bitstring} package (\texttt{pip install bitstring})  \item Approximately 1ms computation time per prediction\end{itemize}All dependencies are minimal and widely available, ensuring maximum reproducibility across different computing environments.\subsection{Version Control and Reproducibility}For full version-controlled scripts, input/output data, and complete documentation, see the GitHub repository at \url{https://github.com/Bigrob7605/MMH}. All code, data, and analysis pipelines are publicly available for verification and extension. before the bibliography.\clearpage\appendix% Best-practice: improve line breaking in paragraphs\sloppy% Fix section numbering for appendix\renewcommand{\thesection}{A\arabic{section}}\section{Multi-epoch Meta-Hash (MMH) Recursion Framework}\label{app:MMH}The Multi-epoch Meta-Hash (MMH) framework provides a deterministic algorithm mapping a 128-bit binary seed $S$ to three quantized late-Universe cosmological observables:\[  \left(z_{\mathrm{jump}},\, r_{\mathrm{dip}},\, \ell_{\mathrm{CIB}}\right) = \mathrm{MMH}(S)\]This mapping is fully specified, reproducible, and reference code is available at \url{https://github.com/Bigrob7605/MMH}.\subsection{Seed Expansion via SHA-256}Given $S \in \{0,1\}^{128}$, form a 256-bit hash:\[  H = \mathrm{SHA256}(S) \in \{0,1\}^{256}\]Let $H = b_1b_2 \ldots b_{256}$ (bits). Partition as:\[  \begin{aligned}    C_1 &= b_1 \ldots b_{84} \\    C_2 &= b_{85} \ldots b_{168} \\    C_3 &= b_{169} \ldots b_{256}  \end{aligned}\]Each chunk $C_i$ is treated as a big-endian integer in $[0,\,2^{|C_i|}-1]$.\subsection{Quantization to Physical Observables}Map each chunk $C_i$ to a real observable $X$ by:\[  X = X_{\min} + \frac{\mathrm{int}(C_i)}{2^{|C_i|}-1}\,(X_{\max} - X_{\min})\]where the $X_{\min}$ and $X_{\max}$ are set by the physical search windows:\begin{itemize}  \item \textbf{Quantized Expansion Jump:}    \[      z_{\rm jump} = 0.0723 + \frac{C_1}{2^{84}-1} \times 0.0028    \]    With corresponding $\Delta H_0 = 5.73 \pm 0.44~\mathrm{km\,s^{-1}\,\mathrm{Mpc}^{-1}}$.  \item \textbf{Galaxy Clustering Dip:}    \[      r_{\rm dip} = 153.2 + \frac{C_2}{2^{84}-1} \times 1.9~\mathrm{Mpc}\,h^{-1}    \]    Predicts a $3.4\%$ deficit in $\xi(r)$ at this scale.  \item \textbf{CIB--H$_0$ Cross-Correlation:}    \[      \ell_{\rm CIB} = 197 + \frac{C_3}{2^{88}-1} \times 4    \]    Yields a $2.4\sigma$ CIB $\times$ H$_0$ signal at this multipole.\end{itemize}\textit{Parameter ranges reflect plausible late-Universe anomaly windows based on recent Roman, DESI, and SPHEREx forecasts. The hash partitioning ensures full coverage of each target interval.}\subsection{Reference Python Implementation}For reproducibility, the following pseudocode (Python 3.11+, using \texttt{hashlib} and \texttt{bitstring} for clarity):\begin{lstlisting}[language=Python, frame=single, basicstyle=\ttfamily\footnotesize, numbers=left, numberstyle=\tiny, stepnumber=1, numbersep=5pt, caption=MMH recursion framework implementation, label=lst:mmh_impl]# Python 3.11+; requires bitstring.Bits moduleimport hashlibimport bitstringdef MMH(seed_128bit_hex):    """    Multi-epoch Meta-Hash recursion framework.        Args:        seed_128bit_hex (str): 128-bit hex string (e.g., '0x7f3a2c9e45af01b6da2d4316a2b0e5d1')        Returns:        tuple: (z_jump, r_dip, ell_CIB) with quantized predictions    """    # Convert seed to bytes and hash via SHA-256    seed_bytes = bytes.fromhex(seed_128bit_hex[2:])  # drop '0x'    h256 = hashlib.sha256(seed_bytes).digest()    bits = bitstring.Bits(bytes=h256)        # Extract chunks    C1 = bits[0:84].uint    C2 = bits[84:168].uint    C3 = bits[168:256].uint        # Quantized predictions    z_jump  = 0.0723 + (C1/(2**84-1)) * 0.0028    r_dip   = 153.2  + (C2/(2**84-1)) * 1.9    ell_CIB = 197    + (C3/(2**88-1)) * 4        return round(z_jump,4), round(r_dip,1), round(ell_CIB,0)# Example usage:# MMH("0x7f3a2c9e45af01b6da2d4316a2b0e5d1")# Returns: (0.0723, 153.2, 197)\end{lstlisting}\textbf{Example:}  Input seed: \texttt{0x7f3a2c9e45af01b6da2d4316a2b0e5d1}  Output:\[  z_{\rm jump} \approx 0.0723, \quad r_{\rm dip} \approx 153.2, \quad \ell_{\rm CIB} \approx 197\]\subsection{Rationale and Limitations}The MMH procedure is intentionally minimal, transparent, and easy to reproduce—anyone can re-derive the results. The specific mapping windows reflect prior known anomalies and test regions, but the use of a deterministic cryptographic hash (SHA-256) is for maximal auditability, not as a claim about the physical universe's mechanism.\textbf{Important Disclaimer:} The use of a cryptographic hash is strictly for reproducibility and coverage—not a claim about physical cosmology. SHA-256 is chosen for its deterministic properties and widespread availability, not because it has any physical significance. This framework is a mathematical falsifiability exercise designed for explicit testing, not a theory of cosmic microphysics.This framework is agnostic to the true microphysics: its success or failure is decided solely by the appearance (or non-appearance) of the predicted signatures in Roman, DESI, and SPHEREx data.\textit{Full details, data, and cross-check scripts are versioned in the accompanying public repository.}\subsection{Computational Requirements}The MMH framework requires:\begin{itemize}  \item Python 3.11 or higher  \item \texttt{hashlib} (standard library)  \item \texttt{bitstring} package (\texttt{pip install bitstring})  \item Approximately 1ms computation time per prediction\end{itemize}All dependencies are minimal and widely available, ensuring maximum reproducibility across different computing environments.\subsection{Version Control and Reproducibility}For full version-controlled scripts, input/output data, and complete documentation, see the GitHub repository at \url{https://github.com/Bigrob7605/MMH}. All code, data, and analysis pipelines are publicly available for verification and extension. before the bibliography.\clearpage\appendix% Best-practice: improve line breaking in paragraphs\sloppy% Fix section numbering for appendix\renewcommand{\thesection}{A\arabic{section}}\section{Multi-epoch Meta-Hash (MMH) Recursion Framework}\label{app:MMH}The Multi-epoch Meta-Hash (MMH) framework provides a deterministic algorithm mapping a 128-bit binary seed $S$ to three quantized late-Universe cosmological observables:\[  \left(z_{\mathrm{jump}},\, r_{\mathrm{dip}},\, \ell_{\mathrm{CIB}}\right) = \mathrm{MMH}(S)\]This mapping is fully specified, reproducible, and reference code is available at \url{https://github.com/Bigrob7605/MMH}.\subsection{Seed Expansion via SHA-256}Given $S \in \{0,1\}^{128}$, form a 256-bit hash:\[  H = \mathrm{SHA256}(S) \in \{0,1\}^{256}\]Let $H = b_1b_2 \ldots b_{256}$ (bits). Partition as:\[  \begin{aligned}    C_1 &= b_1 \ldots b_{84} \\    C_2 &= b_{85} \ldots b_{168} \\    C_3 &= b_{169} \ldots b_{256}  \end{aligned}\]Each chunk $C_i$ is treated as a big-endian integer in $[0,\,2^{|C_i|}-1]$.\subsection{Quantization to Physical Observables}Map each chunk $C_i$ to a real observable $X$ by:\[  X = X_{\min} + \frac{\mathrm{int}(C_i)}{2^{|C_i|}-1}\,(X_{\max} - X_{\min})\]where the $X_{\min}$ and $X_{\max}$ are set by the physical search windows:\begin{itemize}  \item \textbf{Quantized Expansion Jump:}    \[      z_{\rm jump} = 0.0723 + \frac{C_1}{2^{84}-1} \times 0.0028    \]    With corresponding $\Delta H_0 = 5.73 \pm 0.44~\mathrm{km\,s^{-1}\,\mathrm{Mpc}^{-1}}$.  \item \textbf{Galaxy Clustering Dip:}    \[      r_{\rm dip} = 153.2 + \frac{C_2}{2^{84}-1} \times 1.9~\mathrm{Mpc}\,h^{-1}    \]    Predicts a $3.4\%$ deficit in $\xi(r)$ at this scale.  \item \textbf{CIB--H$_0$ Cross-Correlation:}    \[      \ell_{\rm CIB} = 197 + \frac{C_3}{2^{88}-1} \times 4    \]    Yields a $2.4\sigma$ CIB $\times$ H$_0$ signal at this multipole.\end{itemize}\textit{Parameter ranges reflect plausible late-Universe anomaly windows based on recent Roman, DESI, and SPHEREx forecasts. The hash partitioning ensures full coverage of each target interval.}\subsection{Reference Python Implementation}For reproducibility, the following pseudocode (Python 3.11+, using \texttt{hashlib} and \texttt{bitstring} for clarity):\begin{lstlisting}[language=Python, frame=single, basicstyle=\ttfamily\footnotesize, numbers=left, numberstyle=\tiny, stepnumber=1, numbersep=5pt, caption=MMH recursion framework implementation, label=lst:mmh_impl]# Python 3.11+; requires bitstring.Bits moduleimport hashlibimport bitstringdef MMH(seed_128bit_hex):    """    Multi-epoch Meta-Hash recursion framework.        Args:        seed_128bit_hex (str): 128-bit hex string (e.g., '0x7f3a2c9e45af01b6da2d4316a2b0e5d1')        Returns:        tuple: (z_jump, r_dip, ell_CIB) with quantized predictions    """    # Convert seed to bytes and hash via SHA-256    seed_bytes = bytes.fromhex(seed_128bit_hex[2:])  # drop '0x'    h256 = hashlib.sha256(seed_bytes).digest()    bits = bitstring.Bits(bytes=h256)        # Extract chunks    C1 = bits[0:84].uint    C2 = bits[84:168].uint    C3 = bits[168:256].uint        # Quantized predictions    z_jump  = 0.0723 + (C1/(2**84-1)) * 0.0028    r_dip   = 153.2  + (C2/(2**84-1)) * 1.9    ell_CIB = 197    + (C3/(2**88-1)) * 4        return round(z_jump,4), round(r_dip,1), round(ell_CIB,0)# Example usage:# MMH("0x7f3a2c9e45af01b6da2d4316a2b0e5d1")# Returns: (0.0723, 153.2, 197)\end{lstlisting}\textbf{Example:}  Input seed: \texttt{0x7f3a2c9e45af01b6da2d4316a2b0e5d1}  Output:\[  z_{\rm jump} \approx 0.0723, \quad r_{\rm dip} \approx 153.2, \quad \ell_{\rm CIB} \approx 197\]\subsection{Rationale and Limitations}The MMH procedure is intentionally minimal, transparent, and easy to reproduce—anyone can re-derive the results. The specific mapping windows reflect prior known anomalies and test regions, but the use of a deterministic cryptographic hash (SHA-256) is for maximal auditability, not as a claim about the physical universe's mechanism.\textbf{Important Disclaimer:} The use of a cryptographic hash is strictly for reproducibility and coverage—not a claim about physical cosmology. SHA-256 is chosen for its deterministic properties and widespread availability, not because it has any physical significance. This framework is a mathematical falsifiability exercise designed for explicit testing, not a theory of cosmic microphysics.This framework is agnostic to the true microphysics: its success or failure is decided solely by the appearance (or non-appearance) of the predicted signatures in Roman, DESI, and SPHEREx data.\textit{Full details, data, and cross-check scripts are versioned in the accompanying public repository.}\subsection{Computational Requirements}The MMH framework requires:\begin{itemize}  \item Python 3.11 or higher  \item \texttt{hashlib} (standard library)  \item \texttt{bitstring} package (\texttt{pip install bitstring})  \item Approximately 1ms computation time per prediction\end{itemize}All dependencies are minimal and widely available, ensuring maximum reproducibility across different computing environments.\subsection{Version Control and Reproducibility}For full version-controlled scripts, input/output data, and complete documentation, see the GitHub repository at \url{https://github.com/Bigrob7605/MMH}. All code, data, and analysis pipelines are publicly available for verification and extension. before the bibliography.\clearpage\appendix% Best-practice: improve line breaking in paragraphs\sloppy% Fix section numbering for appendix\renewcommand{\thesection}{A\arabic{section}}\section{Multi-epoch Meta-Hash (MMH) Recursion Framework}\label{app:MMH}The Multi-epoch Meta-Hash (MMH) framework provides a deterministic algorithm mapping a 128-bit binary seed $S$ to three quantized late-Universe cosmological observables:\[  \left(z_{\mathrm{jump}},\, r_{\mathrm{dip}},\, \ell_{\mathrm{CIB}}\right) = \mathrm{MMH}(S)\]This mapping is fully specified, reproducible, and reference code is available at \url{https://github.com/Bigrob7605/MMH}.\subsection{Seed Expansion via SHA-256}Given $S \in \{0,1\}^{128}$, form a 256-bit hash:\[  H = \mathrm{SHA256}(S) \in \{0,1\}^{256}\]Let $H = b_1b_2 \ldots b_{256}$ (bits). Partition as:\[  \begin{aligned}    C_1 &= b_1 \ldots b_{84} \\    C_2 &= b_{85} \ldots b_{168} \\    C_3 &= b_{169} \ldots b_{256}  \end{aligned}\]Each chunk $C_i$ is treated as a big-endian integer in $[0,\,2^{|C_i|}-1]$.\subsection{Quantization to Physical Observables}Map each chunk $C_i$ to a real observable $X$ by:\[  X = X_{\min} + \frac{\mathrm{int}(C_i)}{2^{|C_i|}-1}\,(X_{\max} - X_{\min})\]where the $X_{\min}$ and $X_{\max}$ are set by the physical search windows:\begin{itemize}  \item \textbf{Quantized Expansion Jump:}    \[      z_{\rm jump} = 0.0723 + \frac{C_1}{2^{84}-1} \times 0.0028    \]    With corresponding $\Delta H_0 = 5.73 \pm 0.44~\mathrm{km\,s^{-1}\,\mathrm{Mpc}^{-1}}$.  \item \textbf{Galaxy Clustering Dip:}    \[      r_{\rm dip} = 153.2 + \frac{C_2}{2^{84}-1} \times 1.9~\mathrm{Mpc}\,h^{-1}    \]    Predicts a $3.4\%$ deficit in $\xi(r)$ at this scale.  \item \textbf{CIB--H$_0$ Cross-Correlation:}    \[      \ell_{\rm CIB} = 197 + \frac{C_3}{2^{88}-1} \times 4    \]    Yields a $2.4\sigma$ CIB $\times$ H$_0$ signal at this multipole.\end{itemize}\textit{Parameter ranges reflect plausible late-Universe anomaly windows based on recent Roman, DESI, and SPHEREx forecasts. The hash partitioning ensures full coverage of each target interval.}\subsection{Reference Python Implementation}For reproducibility, the following pseudocode (Python 3.11+, using \texttt{hashlib} and \texttt{bitstring} for clarity):\begin{lstlisting}[language=Python, frame=single, basicstyle=\ttfamily\footnotesize, numbers=left, numberstyle=\tiny, stepnumber=1, numbersep=5pt, caption=MMH recursion framework implementation, label=lst:mmh_impl]# Python 3.11+; requires bitstring.Bits moduleimport hashlibimport bitstringdef MMH(seed_128bit_hex):    """    Multi-epoch Meta-Hash recursion framework.        Args:        seed_128bit_hex (str): 128-bit hex string (e.g., '0x7f3a2c9e45af01b6da2d4316a2b0e5d1')        Returns:        tuple: (z_jump, r_dip, ell_CIB) with quantized predictions    """    # Convert seed to bytes and hash via SHA-256    seed_bytes = bytes.fromhex(seed_128bit_hex[2:])  # drop '0x'    h256 = hashlib.sha256(seed_bytes).digest()    bits = bitstring.Bits(bytes=h256)        # Extract chunks    C1 = bits[0:84].uint    C2 = bits[84:168].uint    C3 = bits[168:256].uint        # Quantized predictions    z_jump  = 0.0723 + (C1/(2**84-1)) * 0.0028    r_dip   = 153.2  + (C2/(2**84-1)) * 1.9    ell_CIB = 197    + (C3/(2**88-1)) * 4        return round(z_jump,4), round(r_dip,1), round(ell_CIB,0)# Example usage:# MMH("0x7f3a2c9e45af01b6da2d4316a2b0e5d1")# Returns: (0.0723, 153.2, 197)\end{lstlisting}\textbf{Example:}  Input seed: \texttt{0x7f3a2c9e45af01b6da2d4316a2b0e5d1}  Output:\[  z_{\rm jump} \approx 0.0723, \quad r_{\rm dip} \approx 153.2, \quad \ell_{\rm CIB} \approx 197\]\subsection{Rationale and Limitations}The MMH procedure is intentionally minimal, transparent, and easy to reproduce—anyone can re-derive the results. The specific mapping windows reflect prior known anomalies and test regions, but the use of a deterministic cryptographic hash (SHA-256) is for maximal auditability, not as a claim about the physical universe's mechanism.\textbf{Important Disclaimer:} The use of a cryptographic hash is strictly for reproducibility and coverage—not a claim about physical cosmology. SHA-256 is chosen for its deterministic properties and widespread availability, not because it has any physical significance. This framework is a mathematical falsifiability exercise designed for explicit testing, not a theory of cosmic microphysics.This framework is agnostic to the true microphysics: its success or failure is decided solely by the appearance (or non-appearance) of the predicted signatures in Roman, DESI, and SPHEREx data.\textit{Full details, data, and cross-check scripts are versioned in the accompanying public repository.}\subsection{Computational Requirements}The MMH framework requires:\begin{itemize}  \item Python 3.11 or higher  \item \texttt{hashlib} (standard library)  \item \texttt{bitstring} package (\texttt{pip install bitstring})  \item Approximately 1ms computation time per prediction\end{itemize}All dependencies are minimal and widely available, ensuring maximum reproducibility across different computing environments.\subsection{Version Control and Reproducibility}For full version-controlled scripts, input/output data, and complete documentation, see the GitHub repository at \url{https://github.com/Bigrob7605/MMH}. All code, data, and analysis pipelines are publicly available for verification and extension.